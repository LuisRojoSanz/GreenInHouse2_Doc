\apendice{Plan de Proyecto Software}

\section{Introducción}
En este apéndice se desarrolla el plan de proyecto seguido para la creación de la aplicación de GreenInHouse 2.0, tratando la planificación temporal y el análisis de viabilidad del proyecto. 
En la planificación temporal se detalla la organización que ha seguido el proyecto con los diferentes sprints en los que se ha dividido y en lo desarrollado en cada uno de ellos.
Por otro lado, también se ha llevado a cabo un estudio de la viabilidad para analizar los factores que han influido en el desarrollo de la aplicación. Se ha estudiado la viabilidad económica, evaluando los recursos necesarios y costos junto con la viabilidad legal asegurando que cumple con la normativa vigente.
Este plan de proyecto nos permite garantizar que aunque el proyecto sea posible a nivel técnico, también lo tiene que ser a nivel económico y legal.

\section{Planificación temporal}
La planificación temporal de este proyecto se ha basado en la realización de un sprint cada dos semanas reuniéndome con el tutor del proyecto Raúl Marticorena, con Yeray Pescador y con Oscar Valverde para ver los cambios realizadas durante ese tiempo, viendo si dichos cambios estaban bien implementados y proponiendo nuevas tareas a realizar.

\subsection{Sprint 0 Planificación y Diseño Inicial 10/10/2024 - 24/10/2024}
Durante el primer sprint se llevó a cabo la planificación y un pequeño diseño de la aplicación. El diseño lo realicé creando un \textit{mockup} de las principales pantallas que tendría la aplicación como son la de inicio, la de gráficos, la de hitos, la de creación de plantas y la de ajustes. Esto lo llevé a cabo mediante el uso de una herramienta llamada  ``Pencil''.
Se investigaron las herramientas que iba a usar en la creación de la aplicación midiendo entre otras cosas, la compatibilidad con el uso de APIs para poder consultar los datos que hay guardados en la base de datos del backend. Llegamos a la conclusión que la mejores herramientas para este fin era usar Flutter y Dart integradas ya en Android Studio. 
Por último instalé Android Studio descargando Flutter, Dart y diferentes librerías necesarias ya que si no me daban ciertos errores durante la instalación.

\subsection{Sprint 1 Desarrollo Inicial de Pantallas 24/10/2024 - 07/11/2024}
Durante este segundo sprint comencé con el desarrollo de la aplicación creando las primeras pantallas. Implementé la pantalla de inicio, en la que añadí un panel inferior donde más adelante iría implementando las referencias a otras pantallas. También creé una primera versión de la pantalla de creación de plantas aunque todavía no era del todo funcional. En ambas pantallas incorporé la función de internacionalización. Por último conseguí que la integración de la API de la raspberry con el Android Studio fuera funcional.

\subsection{Sprint 2 Creación y Mejora de Pantallas 07/11/2024 - 21/11/2024}
Este tercer sprint se basó sobre todo en la mejora de pantallas ya existentes y creación de otras. Creé las pantallas de cambio de idioma, consejos plantas y la de comprobación de sensores implementando la función de internacionalización de ellas. Además mejoré algunos detalles de la pantalla de inicio.

\subsection{Sprint 3 Mejoras Visuales de las Pantallas  21/11/2024 - 05/12/2024}
Este cuarto sprint se basó en la mejora de pantallas que ya había creado previamente. Mejoré la interfaz de la pantalla de cambio de idioma implementando iconos de banderas para que fuera mas visual para el usuario. Además, modifiqué tanto la pantalla de consejos de plantas y la pantalla de los sensores para que fuera también mas visual e intuitiva para el usuario mediante el uso de iconos y colores.


\subsection{Sprint 4 Implementación de Gráficas y Visualización de Datos 05/12/2024 - 19/12/2024}
Este quinto sprint se basó en la creación de una gráfica para hacer más visual la representación de los datos que llegan desde la API. Creé la pantalla de gráficos e implementé un gráfico para ver como se representaban los datos en el mismo para luego poder trabajar mejor con ellos. Además, probé a usar el método "POST" para poder enviar datos desde la aplicación a la API y que quedaran guardados en la base de datos.

\subsection{Sprint 5 Optimización de Gráficas y Refinamiento de la Interfaz  19/12/2024 - 23/1/2025}
Durante este sexto sprint estuve mirando como mejorar el gráfico y crearlo de una manera que fuera muchos mas visual e intuitiva. Además, creé tanto la pantalla de eliminar plantas como la de modificar plantas para que la aplicación contemplase dichas funcionalidades.
Por último estuve intentando adaptar la pantalla al poner el móvil en horizontal.

\subsection{Sprint 6 Creación de nuevas Gráficas y mejora de la Documentación  23/1/2025 - 6/2/2025}
Este séptimo sprint me centré en la creación tanto de la gráfica de luz como la de temperatura. Además, he seguido mejorando el conjunto de todas ellas para que cada vez vayan quedando mas visible para el usuario.
También me centré en hacer los apartados del plan de proyecto y el de la especificación de requisitos de la documentación.

\subsection{Sprint 7 Creación de la pantalla de Hitos y de la Gráfica de Humedad de Ambiente  6/2/2025 - 27/2/2025}
Durante este octavo sprint me centré en la creación de varios hitos diarios relacionados con los sensores tanto de humedad, temperatura y luz a completar diariamente.
A parte, agregué la gráfica de humedad de ambiente a las ya creadas.

\subsection{Sprint 8 Mejora en los Hitos y Gráficas 27/2/2024 - 13/3/2025}
Durante este noveno sprint me centré sobre todo en la mejora de los hitos. Hice cambios para que la pantalla fuera más visual y mas e interactiva para el usuario separando los hitos temporalmente dependiendo de cada cuanto se tenían que completar dichos hitos. Además, agregué un nuevo hito mensual que requiere de la interacción del usuario con la aplicación para que aparezca como completado.
También hice alguna pequeña mejora en las gráficas como fueron el cambio de un icono y cambios en los límites de las bandas de las gráficas.

\subsection{Sprint 9  13/3/2024 - 27/3/2025}


\subsection{Sprint 10  27/3/2024 - 10/4/2025}


\subsection{Sprint 11  10/4/2024 - 24/4/2025}


\section{Estudio de viabilidad}
La viabilidad del proyecto debe ser evaluado tanto a partir del ámbito económico como del legal ya que aunque un proyecto sea parezca perfecto, puede fracasar si los costes son inviables o si no cumple con la normativa legal estipulada.
Por ello se necesita que un proyecto pueda ser sostenible en el tiempo y acate la normativa actual.

En este apartado se realizará un estudio detallado de la viabilidad económica analizando los costes que supondría el desarrollo del proyecto y formas de retorno de la inversión. A parte, se hará un estudio de la viabilidad legal del proyecto detallando las normativas actuales y licencias de herramientas usadas principalmente.

Ambos estudios serán explicados con ejemplos realistas para demostrar que es un proyecto viable tanto económico como legalmente.

\subsection{Viabilidad económica}
En este apartado se analizará la viabilidad económica del proyecto desde dos perspectivas diferentes. 

En primero lugar, se estudiará la viabilidad del desarrollo llevado a cabo por el alumno donde se explicarán los costes reales empleados para el desarrollo de la aplicación. 

Seguido, se estudiará la viabilidad del proyecto orientado a un proceso de comercialización estudiando su sostenibilidad y viabilidad.



\subsubsection{Costes de desarrollo}

Se van a explicar los costes asociados tanto al software y hardware junto con los costes de personal que se necesita para la realización del proyecto:

\subsubsection{Coste Hardware}
En la tabla \ref{tab:CostosHardware} aparecen reflejados los gastos que se han llevado a cabo tanto para el desarrollo de la aplicación, como para realizar las pruebas necesarias hasta su correcto funcionamiento.

\begin{table}[H]
\centering
\begin{tabular}{|l|c|}
\hline
\textbf{Componente} & \textbf{Coste (€)} \\ \hline
Ordenador portátil & 1000 \\ \hline
Teléfono móvil & 250 \\ \hline
\textbf{Total} & \textbf{1250} \\ \hline
\end{tabular}
\caption{Costes hardware}
\label{tab:CostosHardware}
\end{table}

\subsubsection{Coste Software}
En cuanto al Software y licencias, no ha sido necesario hacer ningún gasto debido a que todas las herramientas usadas en el desarrollo como pueden ser Android Studio o Flutter han sido completamente gratis.

El costo total ha sido de cero euros por lo que se demuestra que se pueden realizar proyectos sin la necesidad de tener casi gastos en software por la amplia variedad de herramientas gratuitas que existen.


\subsubsection{Coste Desarrollador}
Aunque la aplicación la ha llevado a cabo el propio alumno, se ha planteado un coste ficticio que debería haber costado la realización de la misma.

Se ha planteado un escenario en el que el desarrollador trabaja a media jornada durante unos 8 meses que es el tiempo que dura el proyecto. Durante ese tiempo el desarrollador al ser un programador junior cobrará un salario bruto de 900 euros al mes.

A continuación, se puede ver en la tabla \ref{tab:CostosPersonal} los gastos totales al contratar al desarrolador.

\begin{table}[H]
\centering
\begin{tabular}{|l|c|}
\hline
\textbf{Sección} & \textbf{Coste (€)} \\ \hline
Coste mensual para la empresa & 1170 \\ \hline
Retención IRPF (12\%) & 108  \\ \hline
Cotización a la Seguridad Social (30\%) & 270 \\ \hline
Salario neto mensual & 792 \\ \hline
\textbf{Salario neto tras los 8 meses } & \textbf{9360} \\ \hline
\end{tabular}
\caption{Costes Desarrollador}
\label{tab:CostosPersonal}
\end{table}


\subsubsection{Coste Total}

En la tabla \ref{tab:CosteTotal} se muestra la suma de todos los gastos necesarios para el desarrollo del proyecto.


\begin{table}[H]
\centering
\begin{tabular}{|l|c|}
\hline
\textbf{Sección} & \textbf{Coste (€)} \\ \hline
Coste Hardware & 1250 \\ \hline
Coste Desarrollador & 9360 \\ \hline
\textbf{Coste Total} & \textbf{10610} \\ \hline
\end{tabular}
\caption{Coste Total}
\label{tab:CosteTotal}
\end{table}


\subsubsection{Costes de comercialización}

Se van a explicar los costes asociados tanto al software y hardware junto con los costes de personal que se necesita para la comercialización de la aplicación.

Al ser un proyecto de desarrollo de una aplicación, solo se tendrán en cuenta los gastos asociados a dicha actividad. No se tendrán en cuenta gastos como pueden ser las raspberry pi o la fabricación de las macetas.

En este escenario planteado se van a calcular los costes únicamente teniendo un solo desarrollador. Los costes tanto hardware como de desarrollador aumentarían proporcionalmente en función de cuántos desarrolladores más se contrataran. Es decir, si se contratara a otro desarrolador más los costes de hardware y de desarrollador se duplicarían.

\subsubsection{Coste Hardware Comercialización}
En la tabla \ref{tab:CostosHardwareComer} aparecen reflejados los gastos que se han llevado a cabo tanto para el desarrollo de la aplicación, como para realizar las pruebas necesarias hasta su correcto funcionamiento.

\begin{table}[H]
\centering
\begin{tabular}{|l|c|}
\hline
\textbf{Componente} & \textbf{Coste (€)} \\ \hline
Ordenador portátil & 1000 \\ \hline
Teléfono móvil & 250 \\ \hline
\textbf{Total} & \textbf{1250} \\ \hline
\end{tabular}
\caption{Costes Hardware Comercialización}
\label{tab:CostosHardwareComer}
\end{table}


\subsubsection{Coste Software Comercialización}
En cuanto al Software y licencias, han sido necesarios pocos gastos debido a que las herramientas usadas en el desarrollo de la aplicación han sido gratuitas.
Estos pequeños gastos se han reflejado en la tabla \ref{tab:CostosSoftwareCom}

\begin{table}[H]
\centering
\begin{tabular}{|l|c|}
\hline
\textbf{Sección} & \textbf{Coste (€)} \\ \hline
Cuenta de desarrollador de Google Play & 25 \\ \hline
Cuenta de desarrollador de App Store & 99  \\ \hline
Dominio web para la marca & 12 \\ \hline
\textbf{Gasto Software total} & \textbf{136} \\ \hline
\end{tabular}
\caption{Costes Desarrollador}
\label{tab:CostosSoftwareCom}
\end{table}


\subsubsection{Coste Desarrollador Comercialización}
Se ha planteado un escenario en el que se va a contratar a un profesional con 3 años de experiencia para que trabaje a jornada completa y pueda mejorar y dar soporte a la aplicación.

En la tabla \ref{tab:CostosPersonalCom} se va a calcular los gastos de los primeros 6 meses que esté contratado el desarrollador que corresponden con los meses que va a estar en periodo de prueba.

\begin{table}[H]
\centering
\begin{tabular}{|l|c|}
\hline
\textbf{Sección} & \textbf{Coste (€)} \\ \hline
Coste mensual para la empresa & 2751 \\ \hline
Retención IRPF (15\%) & 315  \\ \hline
Cotización a la Seguridad Social (31\%) & 651 \\ \hline
Salario neto mensual & 1785 \\ \hline
\textbf{Salario bruto tras los 6 meses de prueba} & \textbf{16506} \\ \hline
\end{tabular}
\caption{Costes Desarrollador}
\label{tab:CostosPersonalCom}
\end{table}


\subsubsection{Coste Total Comercialización}

En la tabla \ref{tab:CosteTotalCom} se muestra la suma de todos los gastos necesarios de los 6 primeros meses en la parte de la aplicación móvil para la comercialización del producto.

\begin{table}[H]
\centering
\begin{tabular}{|l|c|}
\hline
\textbf{Sección} & \textbf{Coste (€)} \\ \hline
Coste Hardware & 1250 \\ \hline
Coste Software & 136 \\ \hline
Coste Desarrollador & 16506 \\ \hline
\textbf{Coste Total} & \textbf{17892} \\ \hline
\end{tabular}
\caption{Coste 6 meses Comercialización}
\label{tab:CosteTotalCom}
\end{table}


\subsection{Viabilidad legal}

En el desarrollo de GreenInHouse 2.0 se han utilizado aplicaciones y herramientas gratuitas respetando las licencias y condiciones de uso de cada una.

A continuación se van a detallar las aplicaciones y herramientas usadas en el desarrollo:

\begin{itemize}
    \item \textbf{Flutter:} Flutter es un software libre y se distribuye bajo la licencia de tipo BSD la cual permite que se pueda usar con fines comerciales sin ningún coste. Esto hace que Flutter sea una de las mejores opciones para la creación de aplicaciones ya sea con fines educativos o para su futura comercialización.
    \item \textbf{Dart:} Dart es un lenguaje de programación desarrollado por Google y se encuentra bajo la licencia BSD al igual que Flutter. Esta licencia hace que esté disponible de manera gratuita y que además esté permitida su uso para fines comerciales.
    \item \textbf{Android Studio:} Android Studio es un software de uso gratuito tanto para proyectos personales como para temas comerciales. No requiere de adquisición de licencias ni suscripciones por lo que no existe ninguna restricción por usar este software en las creación de nuevas aplicaciones y distribuirlas a través de Google Play o App Store.
    \item \textbf{Recursos visuales y paquetes externos:} Durante el desarrollo de la aplicación se han usado iconos provenientes de bibliotecas libres como \textit{Material Icons} que permiten su uso y comercialización y de paquetes externos obtenidos mediante pub.dev, los cuales tienen licencia BSD.
\end{itemize}

Gracias a la existencia de todo este tipo de herramientas gratuitas que permiten su uso con fines de comercialización se ha podido crear la aplicación de GreenInHouse 2.0 y se podría comercializar sin ningún tipo de problema legal.


