\apendice{Plan de Proyecto Software}

\section{Introducción}
En este apéndice se desarrolla el plan de proyecto seguido para la creación de la aplicación de GreenInHouse2.0, tratando la planificación temporal y el análisis de viabilidad del proyecto. 
En la planificación temporal se detalla la organización que ha seguido el proyecto con los diferentes sprints en los que se ha dividido y en lo desarrollado en cada sprint.
Por otro lado, también se ha llevado a cabo un estudio de la viabilidad para analizar los factores que han influido en el desarrollo de la aplicación. Se ha estudiado la viabilidad económica, evaluando los recursos necesarios y costos junto con la viabilidad legal asegurando que cumple con la normativa vigente.
Este plan de proyecto nos permite garantizar que aunque el proyecto sea posible a nivel técnico, también lo tiene que ser a nivel económico y legal.

\section{Planificación temporal}
La planificación temporal de éste proyecto se ha basado en la realización de un sprint cada dos semanas en el que nos reuníamos para ver los cambios realizadas durante ese tiempo, viendo si dichos cambios estaban bien implementados y proponiendo nuevas tareas a realizar.

\subsection{Sprint 0 Planificación y Diseño Inicial 10/10/2024 - 24/10/2024}
Durante el primer sprint se llevó a cabo la planificación y un pequeño diseño de la aplicación. El diseño lo realicé creando un mockup de las principales pantallas que tendría la aplicación como son la de inicio, la de gráficos, la de hitos, la de creación de plantas y la de ajustes. Esto lo llevé a cabo mediante el uso de una herramienta llamada "Pencil".
Se investigaron las herramientas que iba a usar en la creación de la aplicación midiendo entre otras cosas, la compatibilidad con el uso de APIs para poder consultar los datos que hay guardados en la base de datos del backend. Llegamos a la conclusión que la mejores herramientas para este fin era usar Flutter y Dart integradas ya en Android Studio. 
Por último instalé Android Studio descargando Flutter, Dart y diferentes librerías necesarias ya que sino me daban ciertos errores durante la instalación.

\subsection{Sprint 1 Desarrollo Inicial de Pantallas 24/10/2024 - 07/11/2024}
Durante este segundo sprint comencé con el desarrollo de la aplicación creando las primeras pantallas. Implementé la pantalla de inicio, en la que añadí un panel inferior donde más adelante iría implementando las referencias a otras pantallas. También creé una primera versión de la pantalla de creación de plantas aunque todavía no era del todo funcional. En ambas pantallas incorporé la función de internacionalización. Por último conseguí que la integración de la API de la raspberry con el Android Studio fuera funcional.

\subsection{Sprint 2 Creación y Mejora de Pantallas 07/11/2024 - 21/11/2024}
Este tercer sprint se basó sobre todo en la mejora de pantallas ya existentes y creación de otras. Creé las pantallas de cambio de idioma, consejos plantas y la de comprobación de sensores implementando la función de internacionalización de ellas. Además mejoré algunos detalles de la pantalla de inicio.

\subsection{Sprint 3 Mejoras Visuales de las Pantallas  21/11/2024 - 05/12/2024}
Este cuarto sprint se basó en la mejora de pantallas que ya había creado previamente. Mejoré la interfaz de la pantalla de cambio de idioma implementando iconos de banderas para que fuera mas visual para el usuario. Además, modifiqué tanto la pantalla de consejos de plantas y la pantalla de los sensores para que fuera también mas visual e intuitiva para el usuario mediante el uso de iconos y colores.


\subsection{Sprint 4 Implementación de Gráficas y Visualización de Datos 05/12/2024 - 19/12/2024}
Este quinto sprint se basó en la creación de una gráfica para hacer más visual la representación de los datos que llegan desde la API. Creé la pantalla de gráficos e implementé un gráfico para ver como se representaban los datos en el mismo para luego poder trabajar mejor con ellos. Además, probé a usar el método "POST" para poder enviar datos desde la aplicación a la API y que quedaran guardados en la base de datos.

\subsection{Sprint 5 Optimización de Gráficas y Refinamiento de la Interfaz  19/12/2024 - 23/1/2025}
Durante este sexto sprint estuve mirando como mejorar el gráfico y crearlo de una manera que fuera muchos mas visual e intuitiva. Además, creé tanto la pantalla de eliminar plantas como la de modificar plantas para que la aplicación contemplase dichas funcionalidades.
Por último estuve intentando adaptar la pantalla al poner el móvil en horizontal.


\section{Estudio de viabilidad}

\subsection{Viabilidad económica}

\subsection{Viabilidad legal}

