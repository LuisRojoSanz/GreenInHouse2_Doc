\capitulo{1}{Introducción}

En la actualidad, las tecnologías junto con el Internet de las Cosas se encuentran en constante evolución. Estos continuos avances abren un gran abanico de posibilidades en la automatización de tareas de la vida cotidiana. El uso de sensores conectados a la red local ha sido el factor clave para poder llegar a monitorizar estos diferentes entornos, como pueden ser hogares inteligentes o entornos agrícolas.

Dentro de estos avances tecnológicos surge GreenInHouse2.0, una aplicación móvil orientada a facilitar el cuidado y seguimiento de las plantas en un entorno doméstico. Con ayuda de dispositivos como sensores de humedad, temperatura y luminosidad, la aplicación ayuda al usuario a tener controlados los diferentes parámetros que influyen en el estado de la planta. Todo esto  se realiza a través de una interfaz de usuario muy intuitiva que consta de gráficos en los que se pueden ver cuándo los valores se alejan de las zonas óptimas para la planta. Además, también cuenta con un conjunto de hitos que el usuario deberá completar diariamente para que la planta tenga buen cuidado.

Este proyecto muestra que es posible usar una solución tecnológica automatizada para cosas de la vida cotidiana, como puede ser el cuidado de las plantas. Gracias a todas las opciones de personalización y visualización que ofrece la aplicación, los usuarios pueden conocer mejor los cuidados que deben seguir las plantas para que tengan buena salud. Todo ello está adaptado tanto para perfiles más principiantes en el mundo del cuidado de plantas como para gente más experta.

\section{Estructura de la memoria}
La memoria se estructura de la siguiente manera:
\begin{enumerate}
    \item \textbf{Introducción:} Se incluye una breve descripción, se presentan los objetivos generales del proyecto y se explica la estructura tanto de la memoria como de los anexos.
    \item \textbf{Objetivos del proyecto:} Se definen los propósitos tanto técnicos y funcionales como personales que se han tenido en cuenta para el desarrollo del proyecto.
    \item \textbf{Conceptos teóricos:} Se explican aquellos conceptos teóricos más importantes usados en la realización del proyecto
    \item \textbf{Técnicas y herramientas:} Se explican las diferentes tecnologías usadas para el desarrollo del proyecto.
    \item \textbf{Aspectos relevantes del desarrollo del proyecto:} Se explican las diferentes fases del proyecto junto como cómo se han abordado para el correcto desarrollo del mismo.
    \item \textbf{Trabajos relacionados:} Se analizan diferentes proyectos similares en el ámbito del cuidado automatizado de plantas.
    \item \textbf{Conclusiones y líneas de trabajo futuras:} Se explican los diferentes aprendizajes obtenidos junto con mejoras para implementar en futuras versiones de la aplicación.
\end{enumerate}

\section{Estructura de los anexos}
Los anexos se estructuran de la siguiente manera:
\begin{enumerate}
    \item \textbf{A Plan de proyecto:} Se recoge la planificación temporal, los ``sprints'' realizados y el análisis de viabilidad tanto económica como legal.
    \item \textbf{B Requisitos:} Se recogen los requisitos tanto funcionales como no funcionales, junto con los casos de uso.
    \item \textbf{C Diseño:} Se recogen el diseño de datos, el diseño arquitectónico y el diseño procedimental junto a sus respectivos diagramas.
    \item \textbf{D Manual del programador:} Se recogen la estructura de directorios junto con una guía para poder continuar con el desarrollo y mantenimiento de la aplicación.
    \item \textbf{E Manual de usuario:} Se recoge una muestra del funcionamiento de la aplicación para el usuario.
\end{enumerate}