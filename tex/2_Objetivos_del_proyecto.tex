\capitulo{2}{Objetivos del proyecto}

El objetivo principal de este proyecto ha sido el de desarrollar una aplicación que permita que el cuidado de las plantas por parte de los usuarios se maneje de manera más eficiente, gracias a la posibilidad de monitorizar su cuidado con la ayuda de los sensores.

Para llevar a cabo este objetivo principal se han seguido tres tipos diferentes de objetivos:

\section{Objetivos Software:}
\begin{enumerate}
    \item {Permitir la creación, modificación y eliminación de plantas junto con sus respectivos parámetros asociados.}
    \item {Poder visualizar los diferentes tipos de valores registrados por los sensores, tanto de humedad como temperatura y luminosidad.}
    \item {Poder visualizar en los gráficos un histórico de los datos, seleccionado el rango de fechas deseado}
    \item {Mostrar hitos tanto diarios como mensuales relacionados con el cuidado de la planta creada, como pueden ser el cambio de tierra o agregar fertilizante.}
    \item {Poder configurar el idioma, la foto de la planta y la visualización tanto de las gráficas como de los hitos en función de los que desee el usuario.}
    \item {Mostrar consejos asociados a cada tipo de planta un función de la planta que cree el usuario.}
    \item {Mostrar el estado de cuidado de la planta en forma de porcentaje.}
    \item {Mostrar el estado de los sensores}
\end{enumerate}

\section{Objetivos Técnicos:}
\begin{enumerate}
    \item {Desarrollar la aplicación usando Flutter y Dart por su compatibilidad con AndroidStudio y ofrecer buen rendimiento.}
    \item {Garantizar una buena comunicación entre la aplicación y el ``backend'' mediante peticiones HTTP dentro de la red local.}
    \item {Implementar una arquitectura que en el futuro pueda ser lo mas escalable posible para poder seguir mejorando la aplicación o agregar nuevas funcionalidades, como la agregación de nuevos sensores.}
    \item {Usar herramientas de almacenamiento local, como por ejemplo ``SharedPreferences'' y de tratamiento de datos como pueden ser las gráficas.}
    \item {Hacer robusta la aplicación frente a posibles errores que pueda dar la aplicación, como por ejemplo por falta de conexión.}
\end{enumerate}


\section{Objetivos Personales:}
\begin{enumerate}
    \item {Aprender a programar con Flutter y Dart ya que son herramientas punteras en el ámbito de creación de aplicaciones multiplataforma.}
    \item {Aprender a llevar de la forma más óptima tanto la planificación como la organización de un proyecto creado desde cero.}
    \item {Aprender a integrar hardware con software, mediante el uso de peticiones HTTP dentro de la red local.}
    \item {Mejorar la manera de documentar todo el proceso de creación de un proyecto}
\end{enumerate}

